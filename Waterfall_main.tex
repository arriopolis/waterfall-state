\documentclass{article}
%\usepackage[T1]{fontenc}
\usepackage{amsmath}
\usepackage{fancyhdr}
\usepackage{amssymb}
\usepackage{amsthm}
\usepackage{graphicx}
\usepackage{varioref}
\usepackage{verbatim} 
\usepackage{multicol}
\usepackage{enumerate}
%\usepackage[normalem]{ulem}

\usepackage{caption}
\usepackage{subcaption}
%\usepackage[T1]{fontenc}
\usepackage[margin=1in]{geometry}
\usepackage{qcircuit}

%\usepackage{mathrsfs}

\usepackage{url}\urlstyle{same}
\usepackage{xspace}
\usepackage{thm-restate}

% Nicer font
%\usepackage{mathpazo}

% Microtype
\usepackage{microtype}

% TikZ
\usepackage{tikz}
\usetikzlibrary{calc}

% Support eps figures
\usepackage{epstopdf}

% Hypertext package
\usepackage[colorlinks = true]{hyperref}
% Title and authors
%\hypersetup{
%  pdftitle = {},
%  pdfauthor = {}
%}
% Color definitions
\usepackage{xcolor}
\definecolor{darkred}  {rgb}{0.5,0,0}
\definecolor{darkblue} {rgb}{0,0,0.5}
\definecolor{darkgreen}{rgb}{0,0.5,0}
% Color links
\hypersetup{
  urlcolor   = blue,         % color of external links
  linkcolor  = darkblue,     % color of internal links
  citecolor  = darkgreen,    % color of links to bibliography
  filecolor  = darkred       % color of file links
}

% Clever references
\usepackage{cleveref}%[nameinlink]
\crefname{lemma}{Lemma}{Lemmas}
\crefname{proposition}{Proposition}{Propositions}
\crefname{definition}{Definition}{Definitions}
\crefname{theorem}{Theorem}{Theorems}
\crefname{conjecture}{Conjecture}{Conjectures}
\crefname{corollary}{Corollary}{Corollaries}
\crefname{section}{Section}{Sections}
\crefname{appendix}{Appendix}{Appendices}
\crefname{figure}{Fig.}{Figs.}
\crefname{equation}{Eq.}{Eqs.}
\crefname{table}{Table}{Tables}
\crefname{claim}{Claim}{Claims}


%%%%%%%%%%%%%%%%%%%%%%%%%
%  N E W T H E O R E M  %
%%%%%%%%%%%%%%%%%%%%%%%%%

\newtheorem{theorem}{Theorem}
\newtheorem{lemma}[theorem]{Lemma}
\newtheorem{proposition}[theorem]{Proposition}
\newtheorem{definition}[theorem]{Definition}
\newtheorem{corollary}[theorem]{Corollary}
\newtheorem{conjecture}[theorem]{Conjecture}
\newtheorem*{conjecture*}{Conjecture}
\newtheorem*{problem}{Problem}
\newtheorem{claim}[theorem]{Claim}
\theoremstyle{definition}
\newtheorem*{remark}{Remark}
\newtheorem*{example}{Example}


%%%%%%%%%%%%%%%%%%%
%  Comments  %
%%%%%%%%%%%%%%%%%%%
\newcommand{\todo}[1]{{\color{red}{[{\bf TODO:} #1]}}}
\newcommand{\tocite}[1]{{\color{blue}{[{\bf CITE:}#1]}}}
\newcommand{\tocheck}[1]{{\color{red}{[{\bf TO CHECK:}#1]}}}

%%%%%%%%%%%%%%%%%%%
%  Author Affiliations  %
%%%%%%%%%%%%%%%%%%%
%\usepackage{authblk}
%\renewcommand\Authfont{\scshape}
%\renewcommand\Affilfont{\itshape\small}


%%%%%%%%%%%%%%%%%%%
%  New Commands  %
%%%%%%%%%%%%%%%%%%%
\newcommand\QMA{{\sf{QMA}}}
\newcommand\QCMA{{\sf{QCMA}}}
\newcommand\NP{{\sf{NP}}}


\newcommand{\ee}{\mathcal{E}}
\newcommand{\ii}{\mathbb{I}}
\newcommand{\rl}{\rangle\langle}
\newcommand{\mg}{\mathcal{G}}
\newcommand{\hn}[1]{\|#1\|^H_{1\rightarrow 1}}
\newcommand{\ve}[1]{|#1\rangle\!\rangle}
\newcommand{\ro}[1]{\langle\!\langle#1|}
\newcommand{\bkett}[1]{|#1\rangle\!\rangle\langle\!\langle#1|}
\newcommand{\ket}[1]{|#1\rangle}
\newcommand{\bra}[1]{\langle#1|}
\newcommand{\bk}[1]{|#1\rangle\langle#1|}
\newcommand{\tth}[0]{\textsuperscript{th}}
\newcommand{\st}[0]{\textsuperscript{st}}
\newcommand{\nd}[0]{\textsuperscript{nd}}
\newcommand{\rd}[0]{\textsuperscript{rd}}


\DeclareMathAlphabet{\matheu}{U}{eus}{m}{n}

\DeclareMathOperator{\tr}{tr}
\DeclareMathOperator{\id}{id}
\newcommand{\Favg}{\overline{F}}
\newcommand{\Paulis}{{\matheu P}}
\newcommand{\Clifs}{{\matheu C}}
\newcommand{\Hilb}{{\matheu H}}
\newcommand{\T}{{\matheu T}}
\newcommand{\sop}[1]{{\mathcal #1}}
\newcommand{\PL}[1]{{#1}^{P\!L}}
\newcommand{\BHn}{{{\mathcal B}({\mathcal H}^{\otimes n})}}
\newcommand{\I}{{\mathbb I}}
\newcommand{\CNOT}{{\mathrm{CNOT}}}
\newcommand{\plr}[1]{\hat{#1}} %pauli-liouville-represenation


%\newcommand{\ket}[1]{|{#1}\rangle}
%\newcommand{\bra}[1]{\langle{#1}|}
\newcommand{\braket}[2]{\langle{#1}|{#2}\rangle}
\newcommand{\ketbra}[2]{|{#1}\rangle\!\langle{#2}|}
\newcommand{\kket}[1]{|{#1}\rangle\!\rangle}
\newcommand{\bbra}[1]{\langle\!\langle{#1}|}
\newcommand{\bbrakket}[2]{\langle\!\langle{#1}|{#2}\rangle\!\rangle}
\newcommand{\proj}[1]{|#1\rangle\!\langle#1|}

\newcommand{\no}{\nonumber\\}

\newcommand{\ns}{{\textsc{ns}}}
\newcommand{\ceil}[1]{\left\lceil{#1}\right\rceil}
\newcommand{\se}{\succcurlyeq}
\newcommand{\di}{\textrm{diag}}
\newcommand{\capac}{c}
\newcommand{\eps}{\epsilon}

%-----------------------------------------------------------------------------%

%-- Graph Symbols --%
\newcommand{\depth}{d}
\newcommand{\subf}{l}
\newcommand{\edgeL}{\ell}
\newcommand{\Ohm}{c}
\newcommand{\norm}[1]{\left\| #1 \right\|}
\def\O{\mathrm{O}}
\def\tO{\widetilde{\mathrm{O}}}
\renewcommand{\th}[1]{${#1}^{\textrm{th}}$}
\begin{document}
\title{\large {\bf An approximate $\log n$ depth circuit for decoding waterfall states, with aplication to position based cryptography}}
\author{Alvaro Piedrafita, Subhasree Patro, Arjan Corneliessen, Farrokh Labib, Florian Speelman}
\maketitle
\section{Correctness of the circuit}

 Our goal here is to prove that the circuit from figure \ref{fig.1} extracts the value $x_k$ while leaving the state mostly unperturbed. That is, we will prove the following lemma.

\begin{lemma}
Let $\vec{x}=(x_1,\dots,x_k)\in\{0,1\}^k$ be a $k$ bit string, and $\ket{Enc(\vec{x})}$ its encoding.
\begin{equation}
|\bra{Enc(\vec{x})}\bra{x_k}U_k\ket{Enc(\vec{x})}\ket{0}|= \sqrt{1-\frac{\sin^2\frac{\pi}{8}}{2^{k-1}}}
\end{equation}
\end{lemma}
\begin{proof}
We begin by noticing that we can split $U_k$ into $V_k^\dagger CNOT_{(k,A)}V_k$, where $V_k$ are unitaries that only act on the block of $k$ qubits and the controlled not operation acts on the last qubit of the block and the ancilla. The crucial part of the proof will be to understand the structure of  $V_k\ket{Enc(\vec{x})}$. Indeed, allow us to write  $V_k\ket{Enc(\vec{x})}$ as
\[V_k\ket{Enc(\vec{x})}=\ket{\psi}\ket{x_k}+\ket{\phi}\ket{\bar{x_k}},\]
For some vectors $\ket{\psi}$ and $\ket{\phi}$. Observe that the $CNOT$ with a target initialized at $\ket{0}$ simply copies into the ancillary register the value of the $k$-th bit of $V_k\ket{Enc(\vec{x})}$, hence we have
\begin{equation}
CNOT_{(k,A)}V_k\ket{Enc(\vec{x})}\ket{0}=\ket{\psi}\ket{x_k}\ket{x_k}+\ket{\phi}\ket{\bar{x_k}}\ket{\bar{x_k}}.
\end{equation}

Hence, the inner product that we are interested in reads
\begin{eqnarray}
|\bra{Enc(\vec{x})}\bra{x_k}V_k^\dagger CNOT_{(k,A)}V_k\ket{Enc(\vec{x})}\ket{0}|&=&|\left[\left(\bra{\psi}\bra{x_k}+\bra{\phi}\bra{\bar{x_k}}\right)\bra{x_k}\right]\left[\ket{\psi}\ket{x_k}\ket{x_k}+\ket{\phi}\ket{\bar{x_k}}\ket{\bar{x_k}}\right]|\\
&=&|\braket{\psi}{\psi}|=\sqrt{1-|\braket{\phi}{\phi}|^2}.
\end{eqnarray}

Now, we shall characterize $V_k\ket{Enc(\vec{x})}$ and prove that $|\ket{\phi}|$ is really small.
\end{proof}

\end{document}